% !TeX encoding = UTF-8
% !TeX spellcheck = es_ES
% !TeX root = ../../main.tex

\epigraph{Las normas están para cumplirlas, pero cuando se hacen por el beneficio o mejora de todos, si no, es el capricho de alguien que no está trabajando por el colectivo}{Ulises Barrera}

\begin{abstract}
Ponemos unas reglas y disponemos de ellas como una herramienta para facilitarnos el desarrollo de nuestra maqueta. ¿Que nos motiva a tener unas reglas?, ¿Son necesarias?
\end{abstract}

\section{Introducción}
Si bien la cita de Ulises Barrera se refiere a un evento deportivo, ante decisiones arbitrarias de sobre que coches pueden o no disputar una carrera, es un buena explicación de por que existen la reglas, para el beneficio del colectivo y no propio. No sin razón podemos preguntar ¿Que colectivo? si total, la maqueta es para mi mismo y para nadie más.

En el futuro, tendremos que modificar la maqueta, ya sea por mantenimiento o por que queramos ampliarla. El colectivo seremos nuestra versión futura y seguramente no nos acordemos de porque hicimos tal cosa o que cable es el que lleva la alimentación a la vía. Ya que como dice un gran filosofo:

\epigraph{Cuando hice este código solo yo y Dios sabíamos lo que hacia, ahora solo Dios lo sabe}{Comentario anonimo en internet}

Sabias palabras que medio en broma, medio en serio nos muestra la debilidad de nuestra memoria.

\section{Estado del arte}
Hoy por hoy existen muchas normas a la hora de hacer una maqueta. Prácticamente cualquier persona con un blog, canal de youtube o en un foro, expone sus normas, algunos humildemente pero otros de manera tajante. En este apartado trataremos de categorizar y recopilar las normas mas importantes que hemos encontrado.

Las categorías las organizamos según su rango de aplicación, de mas global a más especifica. Dando se la casualidad, que serán de las que menos apliquemos a menos y en caso de seguirlas mal de las que tienen más efecto a menos. Siendo estas:
\begin{itemize}
	\item \textbf{Legislativas}: Las que pone un gobierno o autoridad, que puedan afectar a nuestra maqueta. Suelen ser de seguridad y de sentido común.
	\item \textbf{Para fabricantes}: Son las normas que las asociaciones de fabricantes han puesto para que sus productos sean compatibles, alguna de ellas nos impactara en el diseño.
	\item \textbf{Para Módulos}: Son normas para hacer una maqueta de módulos intercambiables. debemos seguirlas si queremos ir a encuentros y que se pueda unir al resto.
	\item \textbf{Especificas o locales}: Estas las estableceremos para una maqueta en concreto, o si estamos en alguna asociación, las que ponga para poder hacer la maqueta entre varios socios.
\end{itemize}

\subsection{Normativas Legislativas}
El marco legal vigente nos establece una serie de normas en cuanto las actividades que se pueden realizar en según que sitios. No en todos los sitios, aun siendo de nuestra propiedad, podremos construir una maqueta de tren. En general estas normas son de seguridad, y como suele pasar con las leyes sobre seguridad, se ponen tras accidentes donde la gente ha resultado herida. 

Para una maqueta personal y pequeña (de una habitación normal) casi seguro que no haya muchas leyes que nos impacten, mas allá de las normas de convivencia. Aun conviene conocer ciertas normas que nos puedan impactar. 

Conviene conocer las normas que las autoridades locales tengan, podemos ver las siguientes:

\begin{itemize}
	\item \textbf{Reglas de Convivencia}: Básicamente, son el ruido máximo que podemos hacer y en que horas. Pero depende de como queramos "explotar" la maqueta algunas afecten más o menos.
	\item \textbf{Reglas de Construcción}: Aqui tendremos que mirar, si existe alguna ley o normativa que nos indique como debemos construir la maqueta, cuanto puede pesar. En que zonas de una casa. También en este apartado vemos los materiales que se pueden usar o no, por si resultan ser tóxicos en caso de incendio.
	\item \textbf{Reglas Eléctricas}: Puesto que vamos hacer una instalación eléctrica debemos conocer la normativa, para no sobrecargar los conductores. Seguramente simplemente con usar varios enchufes de la habitación sea suficiente.
	\item \textbf{Reglas Sanitarias}: Desde la ventilación que deba tener nuestra habitación, hasta los sanitarios que deba tener.
	\item  \textbf{Normativa de Actividades Económicas}: Si se va realizar una actividad económica en torno a la maqueta es necesario conocerlas. No es el objetivo de estos artículos desarrollar un plan de negocio, y si el lector esta planteándose montar un negocio, seguramente la parte de construcción ya la tenga más que superada.
	
\end{itemize}
Es cierto que esta lista se desarrolla no descartando ninguna para abarcar desde maquetas pequeñas a grandes como "Miniature Wünderland" pasando por el profesional que se dedica a construir maquetas o módulos para otros. Y que por ello muchas normas de esta categoría no se aplicaran, o podemos simplificarlas. También es cierto que ignorarlas (hasta el punto de hacer lo contrario) puede ser fatal.

En general, un maquetista que usa una habitación de su casa, o como mucho un anexo de la casa del pueblo. Solo tiene que preocuparse por no poner materiales peligrosos, no pasarse de peso (para que el suelo no se caiga), de que los cables de luz sean lo suficientemente grandes y de no hacer mucho(pero mucho) ruido por las noches.

Si somos un grupo con un local, deberemos tener en cuenta alguna norma más, como la sanitaria, pero en general, con un conocimiento básico y de sentido común sera suficiente. 

Estas normas alimentaran nuestra normativa de construcción y de explotación.

\subsection{Normativas Para Fabricantes}
En el mundo del modelismo ferroviario hay dos asociaciones de fabricantes, la NMRA de Estados Unidos y NEM de Europa y que a su vez se han coordinado para que sean compatibles y referenciándose entre si.

Estas normas básicamente se establecen para poder correr material de cualquier fabricante sobre maquetas hechas con piezas de diferentes fabricantes. De tal forma podemos usar maquinas de Piko sobre vías Rocco y mezclar coches de varios fabricantes.

Estas normas se dirigen a lo siguiente:
\begin{itemize}
	\item Enganches
	\item Protocolos DCC y LCC
	\item Características eléctricas
	\item Propiedades de las escalas (dimensiones del material)
	\item Distancias de vías (galibo, curvas,...)
\end{itemize}
Para nuestras normativas, realmente solo necesitamos seguir las distancias de vías como una recomendación para ajustarnos a radios para que puedan pasar nuestro material.
\subsection{Normativas Para Módulos}
\subsection{Normativas Especificas o locales}

\section{Experimento}
\section{Texto principal}
\section{Resultados} 
\section{Discusión}
\section{Conclusiones}
\section{Próximos pasos}

\section{Bibliografía}
\printbibliography[heading=subbibliography]
