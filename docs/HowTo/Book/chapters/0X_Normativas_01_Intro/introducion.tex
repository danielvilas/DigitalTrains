% !TeX encoding = UTF-8
% !TeX spellcheck = es_ES
% !TeX root = ../../main.tex

\epigraph{Las normas están para cumplirlas, pero cuando se hacen por el beneficio o mejora de todos, si no, es el capricho de alguien que no está trabajando por el colectivo}{Ulises Barrera}

\begin{abstract}
Ponemos unas reglas y disponemos de ellas como una herramienta para facilitarnos el desarrollo de nuestra maqueta. ¿Que nos motiva a tener unas reglas?, ¿Son necesarias?
\end{abstract}

\section{Introducción}
Si bien la cita de Ulises Barrera se refiere a un evento deportivo, ante decisiones arbitrarias de sobre que coches pueden o no disputar una carrera, es un buena explicación de por que existen la reglas, para el beneficio del colectivo y no propio. No sin razón podemos preguntar ¿Que colectivo? si total, la maqueta es para mi mismo y para nadie más.

En el futuro, tendremos que modificar la maqueta, ya sea por mantenimiento o por que queramos ampliarla. El colectivo seremos nuestra versión futura y seguramente no nos acordemos de porque hicimos tal cosa o que cable es el que lleva la alimentación a la vía.

\epigraph{Cuando hice este código solo yo y Dios sabíamos lo que hacia, ahora solo Dios lo sabe}{Comentario anonimo en internet}

\section{Estado del arte}

\section{Experimento}
\section{Texto principal}
\section{Resultados} 
\section{Discusión}
\section{Conclusiones}
\section{Próximos pasos}

\section{Bibliografía}
\printbibliography[heading=subbibliography]
