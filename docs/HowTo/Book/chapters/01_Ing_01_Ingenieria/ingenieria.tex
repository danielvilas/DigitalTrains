% !TeX encoding = UTF-8
% !TeX spellcheck = es_ES
% !TeX root = ../../main.tex


\epigraph{Su función principal es la de realizar diseños o desarrollar soluciones
    tecnológicas a necesidades sociales, industriales o económicas.
    Para ello el ingeniero debe identificar y comprender los obstáculos más importantes
    para poder realizar un buen diseño. Algunos de los obstáculos son los recursos disponibles,
    las limitaciones físicas o técnicas, la flexibilidad para futuras modificaciones y
    adiciones y otros factores como el coste, la posibilidad de llevarlo a cabo, las
    prestaciones y las consideraciones estéticas y comerciales. Mediante la comprensión de
    los obstáculos, los ingenieros deciden cuáles son las mejores soluciones para afrontar
    las limitaciones encontradas cuando se tiene que producir y utilizar un objeto o sistema.}
{El ingeniero segun Wikipeda}

\epigraph{Alguien que resuelve un problema que no sabías que tenías de una manera
    que no se comprende.}{Chiste anonimo en Internet}

\begin{abstract}
    Veamos que podemos entender por ingenieria y como se puede aplicar a la construccion
    de maquetas. Y quizas podamos llegar a darnos cuenta de que quizas si se estaba aplicando
    ya alguna forma de la misma. Más alla de las tecnicas aplicadas a construccion.
\end{abstract}

\section{Introducción}
Quizas la pregunta más dificil de resolver, y que nos dara un contexto para entender
este capitulo, es definir ``que es la ingenieria'' y con ello podremos ver como aplicarla
en un maqueta.

La otra pregunta a responder, es si ``merece la pena aplicarla'' a una maqueta. Tras varios
puntos de vista, o que se puede entender por ingenieria, comprobraremos que de una forma u
otra ya se estaba haciendo, aunque de forma inconsciente.

Durante la vida profesional del autor se ha encontrado con situaciones (proyectos, clientes,
estudios,...) donde se veia la parte de ingeneria desde diferentes puntos de vista. Iremos
estudiandolos en los diferentes puntos de este partado relacionandolos, cuando sea
posible, con la definicion de Wikipeda.

\subsubsection{Solucionar un/os problema/s}
Tambien se puede decir que \textit{es producir un objeto o sistema}. Es el más obvio, aplicamos
ingenieria para conseguir un objeto, una maqueta en este caso, que antes no teniamos.

Si bien es cierto que la idea que se tiene, es que, solo es para cosas complejas y si es
sencillo no es ingeniria. Es decir debe ser un problema nuevo y complejo y si no lo es,
no es ingenieria.

Pues todo lo contrario, no por ser sencillo o estar ya solucionado no deja de ser ingeniria,
pues el hecho de aprender una solucion de otros y adaptarla a nuestras necesidas ya puede
ser considerado como ingeneria.

Debemos darnos cuenta, que todo gran proyecto, con su problema global, podemos partirlo en
problemillas más pequeños, ``divide y venceras'', y a su vez, nos iremos encontrando con otros
que van surgiendo conforme vayamos avanzando en la construccion de la maqueta.

Desde este punto de vista, el hecho de tener una maqueta al final, cumple para ser ingeneria.

\subsection{Valorar alternivas}
Dentro de la definicion es \textit{decidir cuáles son las mejores soluciones}. Desde esta
prespectiva, ingenieria es pensar en diferentes posiblidades. Ya no solo posibles soluciones,
sino tambien de los posibles problemas que puedan sugir.

Erronenamente hay gente que limita la ingeniria a tener "un" documento que diga hay que hacer
esto que es la mejor opcion de estas planteadas. El error esta en limitarse a esto, ya que en
la relalidad solo es una parte.

Lo más obvio haciendo maquetas, es plantear varios esquemas de vias, y ver cual nos gusta.
Esto en si mismo, ya puede ser considerado como ingeneria, pero para ser más realistas,
ingenieria es llevar al limite esta idea. Como por ejemplo, arboles, comprarlos, hacerlos, si
se hacen, de hilo enrollado, de madera,\dots y asi con todo.

Vemos que desde el momento que plantemos o pensemos en dos opciones para un mismo caso, ya
cumplimos este punto.

\subsection{Proceso}
La realizacion de un proyecto de ingeniria requiere de un proceso bien definido y regulado.
Este proceso garantiza que se \textit{comprenden los obstaculos} y se genera un producto
teniendo en cuenta \textit{los recursos disponibles,las limitaciones físicas o técnicas,
    la flexibilidad  para futuras modificaciones y \dots}.

Estos procesos ademas garantizan que se van alcanzando diferentes hitos y para cada uno se
genera documentacion u otros elementos necesarios para la correcta realizacion del proyecto.
A veces esa documentacion, solo es por motivos regulatorios (nos lo exige alguien, una ley,\dots)
Otras veces son para dar instrucciones a otras personas (planos, lista de materiales, \dots)

Al proceso tambien se le puede llamar metodologia. La idea es que sea algo repetible, metodico
que par nuevos proyectos se aplique ``igual'', para que de esta forma los resultados tengan
calidad\footnote{Habria que definir calidad, pero eso es algo fuera de este capitulo, pues
    la idea se entiende} similar.

A veces que la diferencia entre hacer ingeneria o no, es el proceso. Sobre todo en el universo
DIY o Maker\footnote{DIY: Do It Yourself, o haztelo tu mismo}. Si nos ponemos a hacer cosas
sin una metodologia, sin pensar antes, no es ingeneria. Por ejemplo, saber programar no te hace
ingenerio informatico, hacen falta más cosas. Si siguendo la idea Maker/DIY haces mesas y
no sigues un proceso, cada mesa sera diferente, pero si te defines un proceso, generas
documentacion (medidas, pasos,\dots) podras hacer mesas muy similares y cada vez mejores.

\subsection{Planificacion}
Se dice que un buen proyecto require de una planificacion exquisita, y a veces se piensa que la
ingeneria es hacer una planificacion para que todo el proyecto se haga en el minimo tiempo
posible sin que haya paradas por falta de material, herramientas, \dots.

En nuestra maqueta si un fin de semana no podemos poner las vias porque no tenemos más,
no pasada nada. Pero no se puede parar la construccion de un puente por que nos quedamos
sin cemento. Es labor de alguien preever cuando nos quedaremos sin el y pedir con
antelacion sabiendo lo que se tarda en recibirlo.

Obviamente en nuestra maqueta no debemos ser tan criticos como en un proyecto de ingenieria.
Pero siempre tenemos una lista de tareas por hacer y su secuencia. No es necesario utilizar
herramientas de planing, como el PERT o GANTT\footnote{Formas graficas de ver la dependencia
    entre tareas y su duración}, pero si llevar un pequeño control de tareas o calendario de
cuando pensamos lo que necesitaremos, para no estar parados.

Mientras tengamos un control de los pasos a realizar, tambien cumplimos con este punto.
\subsection{Optimizacion}
Una de las primeras definiciones de ingenieria incluia la optimizacion del uso de los recursos
como su coste economico y temporal. En un lenguaje mas mundano, es minimizar los costes y
maximizar los resultados.

Por ejemplo, imagenemos los tornillos comprados en grandes cantidades, se suelen hacer
descuentos por cantidad. Si nuestro diseño no requiere de sufucientes tornillos para el
siguiente nivel (supongamos que necesitamos 99 y con 100 el precio baja un 25\%) pero estamos
cerca, puede ser interesante añadir ese tornillo que falta y asi rebajar el precio por unidad,
ganando, a priori, mas resitencia al diseño\footnote{Aunque esto dependera de donde se pone y
    algunos factores más.}.

En la practica, la optimizacion en la ingeniria, suele asociarse a planificar para acabar en
el menor tiempo posible y tener el minimo numero de elementos que cumplen los requisitos y sea
seguro. Evitando asi, la sobre-ingeneria.

En una maqueta es tipico ver la optimizacion, en terminos de meter el maximo posible de
elementos en el espacio que tenemos. Ya sean temas de vias, desvios, zonas o escenas  que se
desean. Tambien es escoger el minimo numero de elementos que conforman una escena sin perder
detalle. Un pueblo se puede representar con 2 casas modeladas y un papel de fondo, por ejemplo.
\subsection{Pensar mucho}
El autor ha odido alguna vez que los ignenieros estan todos calvos de tanto pensar, que se les
queman los pelos por el calor de pensar. La verdad es que algun calvo hay, pero como en todos
los trabajos no son tantos y mucho menos como para decir que todos\dots.

De todas formas como dice la definicion es necesario \textit{identificar y comprender los
    obstáculos más importantes}, para ello la forma más facil es haberlos sufrido en anteriodidad
o ponerse a pensar. Pero en la practica lo que hace un ingenerio es pensar antes de actuar.

En nuestras maquetas, algunas veces actuamos sin pensar, nos ponemos a hacer cosas en ella y
si nos gusta, lo dejamos asi, si no nos gusta lo cambiamos. Pero otras pensamos, buscamos
documentacion, fotos, planos de estacion y probamos antes de hacer nada. Ambas opciones son
valida, para una maqueta. Esto no quiere decir que ahora no vamos a probar cosas, ver si nos
gustan o no. Quiere decir que no vamos a pensar antes de probar.

Por ejemplo, no cojemos nuestra caja de vias y desvios para cojer al azar un elemento y
conectarlo con lo que ya tengamos. Por muy creativo que sea este ejercicio, esta muy lejos
del proceso de ingeneria general\footnote{Que bien puede hacerse ante un bloqueo creativo,
    en fases de estudios de alternativas o como pruebas de concepto}.

Pero si lo hacemos en fase de analisis y diseño, partiendo de una idea y cambiando alguna
cosa, si que lo estaria. En este ejemplos, en vez de cojer las vias al azar, podriamos poner
un diseño de playa con 3 vias y ver como añadir 2 más, jugando con las posiciones de los
desvios hasta que nos guste el resultado.
\subsection{Ciencias y tecnicas}
La defincion de la RAE, indica que la ingeneria es aplicar ciencias y tecnicas. Claramente en
al hacer una maqueta se aplican ciencias establecidas y tecnicas concretas. Pero cada ingeneria
requiere de unas ciencias y tecnicas diferentes para cada una, compartiendose entre algunas.

Intentaremos ir recopilando Ciencias y tecnicas a lo largo de diferentes capitulos, que puedan
servir al lector.

\subsection{Finalemente ¿Que es?}
Podemos simplificar la ingeniera en ``Pensar antes de actuar y ser conscientes'', ya que de ahi
se pueden derivar los pilares basicos.
\begin{itemize}
    \item \textbf{Pensar antes de actuar}: O no ir a lo loco, documentarnos de como en la
          realidad se han resuelto los problemas que estamos simulando. O como otros han hecho su maqueta.
          Pensar alternativas y evaluarlas,\dots.
    \item \textbf{Planificar, para el proyecto y para el futuro}: Sin llegar a hacer un plan de
          accion, con todo el analisis que lleva un proyecto de ingeneria. Pero si tener una lista de
          tareas, y pensar que vamos a hacer en el futuro con la maqueta, (ampliarla, hacerla de nuevo,\dots)
    \item \textbf{Realizar un proceso Repetible}: Esto es más importante si vamos a hacer la
          maqueta por modulos\footnote{En este caso secciones que se pueden hacer independiente del resto,
              no solo modulos siguendo algun standard modular}. En cada modulo conviene seguir siempre el mismo
          proceso, para facilitarnos el proceso.
    \item \textbf{Documentar}: para facilitarnos el trabajo en el futuro. Dentro de 10 años no
          nos acordarmos de que cable es que va al carril derecho de la playa de vias. Pero si hemos
          documentado el codigo de colores y el esquema electrico sera más facil.
    \item \textbf{Revisar alternativas}: para tener varias opciones y verificarlas, en linea con
          pensar antes de actuar, veremos una fase de analisis y diseño, cuya mision es esto.
    \item \textbf{Ser conscientes de lo que hacemos}: y para lo que lo hacemos. Es decir tener
          en mente para que hacemos esta "sobrecarga" que es hacer cosas de ignenieros para un "juguete"
    \item \textbf{Ser consecuentes}: Con lo que se decida durante el proyecto y con lo que nos
          obliga a hacer, es decir documentar, que algunas veces podra ser aburrido.
\end{itemize}


\section{Estado del arte}
Explicar como esta actualmente el hobby o las diferentes publicaciones respecto al tema

Maquetas de iniciacion con expansiones

Maquetas hechas por ampliciones

Maquetas Ampliables

Maquetas modulares

Vuelta a empezar

\section{Experimento o Texto principal}
Bien, pensemos en un ejemplo. Hacemos una maqueta y al final tenemos un ovalo con una estacion
y una pequeña playa de vias representando una industria maderera. Pero a partir de tres
aproximaciones diferentes, sin pensar y a lo loco, pensando pero sin seguir un proceso como tal
y por ultimo usando una metodologia de ingenieria.

Este es un ejercicio mental de lo que el autor de este articulo cree que es lo más probable que vaya sucediendo con los fallos apreciados en la maqueta.

\subsection{Recién acabada la maqueta}
Una vez acabada la maqueta en cualquier caso estaremos contentos con nuestra maqueta y en los tres casos tendremos una maqueta muy similar. Para este caso vamos a suponer que los tres casos dan lugar a una maqueta exactamente igual\footnote{En realidad habría pequeñas diferencias, en este apartado queremos estudiar la forma de pensar a largo del tiempo y enseñar que aun siendo iguales, son diferentes} y veremos una serie de cosas particulares para cada caso.
\begin{multicols}{3}
	\textbf{Montaje a lo loco}
	
	Seguramente habrá sido la maqueta más rápida, pues no se habrá parado pensado mucho, quizás algunas pruebas rápidas, pero sin mucho tiempo perdido.
	
	Igualmente habrá varios detalles menores del que pensáramos que seria mejor cambiarlos ligeramente.
	
	\columnbreak
	
	\textbf{Montaje Pensado}
	
	En esta situación, nos habrá llevado más tiempo completar la maqueta, pero al haber pensado lo que queríamos, no tendremos esa sensación de cambiar algunas cosas, en todo caso alguno puntual, pero pocos.
	
	\columnbreak
	
	\textbf{Montaje siguiendo un proceso}
	
	Este caso nos va a llevar mas tiempo, ya que pensamos lo mismo o mas que en el caso anterior y al igual, es posible tener alguna duda sobre algún detalle puntual.
\end{multicols}

\subsection{Después de jugar un tiempo prudencial}
Al pasar el tiempo y después de haber jugado con la maqueta veremos cosas que no nos acaban de gustar del todo, o que mejoraríamos. Este tiempo, podemos definir como lo suficiente para ver los fallos\footnote{Pasar la fase de enamoramiento} y lo suficientemente corto como para recordar nuestras decisiones, si tuviéramos que dar una duración, podríamos decir un año.

En este momento ya podremos ver diferencias de como vemos nuestra maqueta, incluso siendo exactamente iguales.

\begin{multicols}{3}
	\textbf{Montaje a lo loco}
	
	Vamos a ver muchos fallos y cada vez más, nuestro malestar va a empezar a crecer.
	
	Seguramente hagamos algunos cambios, a lo que consideremos lo más grave, pero recordando lo que hicimos y por que, tampoco serán muchos.
	
	\columnbreak
	
	\textbf{Montaje Pensado}
	
	Aun habiendo pensado la maqueta veremos los fallos nuevos, en este momento recordaremos las razones de como se ha llegado a esta maqueta y los aceptaremos tal como son. Quizás, y solo quizás, realicemos algún cambio a alguna cosa  
	
	\columnbreak
	
	\textbf{Montaje siguiendo un proceso}
	
	Este caso, es esencialmente el anterior, pero más pensado y documentado, por lo que seria exactamente lo mismo: Veríamos los fallos, pero recordaríamos o leeríamos las razones y nos quedaríamos igual. Muy probablemente sin cambios.
\end{multicols}
	
Resumiendo, en este momento, para la maqueta pensada y siguiendo un proceso, no tendremos la necesidad (psicológica) de hacer cambios en la maqueta y aceptaremos los fallos tal y como son. 
En todo caso buscaremos cambios menores, modificar ligeramente el trazado de vias, cambiar alguna decoración,... si se hace algun cambio.

Mientras que para la construcción rápida y sin pensar ya empezamos a necesitar cambios. Inicialmente serán menores, pero el cuerpo nos pide más cambios.

\subsection{La madurez de la maqueta}
El tiempo ha seguido pasando, ya no nos acordamos de las razones que nos llevaron a tener la maqueta exactamente así, pero aun no es tan vieja como para desmontarla y hacer una nueva aun le queda vida. Aventurandonos a dar un tiempo nos atrevemos a decir 5 años vista desde la construcción.

En general, los fallos que hemos llegado a aceptar siguen rascándonos la nariz y nuestro malestar va creciendo. No es que no nos divirtamos con la maqueta, simplemente le vamos viendo más fallos. También habremos crecido en el hobby y la maqueta ya no se ajusta a nuestros nuevos intereses. Quizás encontremos que preferíamos material prusiano de época II mientras que la maqueta esta más pensado para material francés de época IV.

Ha llegado un momento critico para la maqueta, nos preguntaremos que haremos a continuación con la maqueta: hacer algún cambio menor, cambio mayor, ampliarla o hacer una de nuevo completamente.


\begin{multicols}{3}
	\textbf{Montaje a lo loco}
	
	Los fallos que vemos cada vez nos molestan más, no sabemos las razones de porque y cada vez las recordamos menos, pero si que fuimos montando-la como quedaba mejor.
	
	El descontento nos pedirá hacer cambios mayores o pensar en hacer una nueva.
	
	A partir de ahora cada vez que juguemos con la maqueta veremos los fallos y el cuerpo nos pedirá ir cambiándola. No tenemos nada, ni recuerdos ni documentación, con los que justificar mantenerlos.
	
	\columnbreak
	
	\textbf{Montaje Pensado}
	
	En este caso ya no nos acordamos de las razones, solo que pensamos algunas variaciones y llegamos a la conclusión de que esta era la mejor. 

	Según nuestra nueva situación nos planteáramos hacer cambios, estos serán grandes en función de nuestras posibilidades. Nuestro descontento ira creciendo, pero como sabemos que todo esta pensado tampoco nos afectara mucho.
	
	Justificaremos los fallos que veamos sabiendo que están ahí y como hemos pensado al crear la maqueta los toleraremos. Siempre nos quedara la duda si no había más opciones y cada vez nos planteamos soluciones que nos guardamos para una futura maqueta.
	
	\columnbreak
	
	\textbf{Montaje siguiendo un proceso}
	
	En esta situación tenemos la documentación con la que hemos ido haciendo la maqueta, por lo que podemos recordar las razones y decisiones que nos han llevado ha este momento. tenemos la certeza de que esta es la mejor opción, aunque despues de unos años podríamos tener más alternativas o nuevas ideas fruto de la experiencia ganada.
	
	Igualmente y según nuestra nueva situación nos planteáramos hacer cambios, estos serán grandes en función de nuestras posibilidades. Nuestro descontento ira creciendo, pero con la documentación justificaríamos los fallos y tampoco nos afectara mucho.
	
	
\end{multicols}
Llegados a este momento la maqueta creada de forma rápida, el cuerpo nos pedirá a gritos un cambio mayor, incluso empezar de nuevo con ella, pero tiene poco tiempo como para que lo veamos como una opción, nos plantearemos cambios mayores (cambiar una parte significativa de las vías, añadir una sección,\dots)

Mientras que para la pensada, seguiremos teniendo la sensación de necesitar el cambio, no tan grande como la versión no pensada, pero si nos plantearemos cambios. La versión documentada, no nos plantearemos cambios, puesto que veremos las alternativas en la documentación y sabremos que ya las habíamos descartado.



\subsection{Los grandes cambios}
Hemos llegado a un momento que vamos hacer grandes cambios, ya sea cambiar drásticamente la maqueta, ampliarla o \dots.

Puede ser por que algunos fallos ya no los toleremos más sin que veamos necesario hacer una maqueta totalmente nueva. Pero lo más probable es que al haber evolucionado en la materia, nuestras necesidades y conocimiento han crecido y la necesidad del cambio en la maqueta sea para actualizarla a nuestros nuevos gustos. O incluso por cambios externos al hobby (más espacio, mudanzas, razones familiares,\dots). 



\begin{multicols}{3}
	\textbf{Montaje a lo loco}
	
	En este caso, la primera idea que nos vendrá a la mente, sera olvidarnos de la maqueta y construir una nueva. Si es que ha llegado a sobrevir hasta este momento\dots
	
	Pero en el caso de hacer grandes cambios, no sabremos como van las cosas. Por ejemplo, los cables que van a las vías serán cada uno de un color, por lo que no sabremos cual va a cual. En una expansión nos tocara ir con el tester probando que puntos conecta cada cable para conectar bien las nuevas vías.
	
	\columnbreak
	
	\textbf{Montaje Pensado}
	
	Seguramente en este caso, nos plateramos hacer cambios o una expansion si nuestras posibilidades lo permiten. Si no se han hecho ya dichos cambios.
	
	En cualquier caso, sera relativamente facil de expandir o cambiar, puesto que se habra llevado una pequeña coordinacion con lo que los diferentes elementos seran identificables, por ejemplo los cables llevaran un esquema de colores, y el carril de la derecha siempre sera del mismo color,\dots
	
	\columnbreak
	
	\textbf{Montaje siguiendo un proceso}
	
	Finalmente, y en esta situacion, seguirmeos el mismo proceso, plantenandos nuevos objetivos y requerimientos. Utilizaremos las lecciones aprendidas, los fallos que hemos encontrado y nos platearemos la primera alternativa, una nueva, expandir o \dots
	
	En cualquier caso tendremos documentacion que nos haran facil reusar los elementos de la maqueta.
\end{multicols}

En resumen, segun que escenario estemos lo tendremos más facil o dificil para avanzar en el hobby.

\subsection{Fallos y Tipos en la maqueta}
En los apartados anteriores hemos hablado de los fallos que vamos a tener en nuestras maquetas, por que efectivamente, todas tendran fallos. Pero el quid de la cuestion, es un fallo para quien juega o para un mero expectador puntual de la misma. 

Empezaramos definiendo un fallo como \emph{aquello que lo vemos y podemos decir ``eso esta mal'' o ``eso nos aburre''}. Bajo esta definicion las unicas personas que pueden decir que es un fallo son el maquetista y las juegan habitualmente con ellas. 

Para entender los fallos, corregirlos y evitarlos en un futuro necesitamos compender cuatro cosas sobre los mismos: que tipo de fallos es, cuando se ha creado, que lo ha causado y que efecto tiene. 

Para el ejercicio mental solo se tiene en cuenta el cuando se han creado, por que es donde puede haber más diferencias segun el estilo de creacion de la maqueta.

Los fallos los podemos clasificar por multidud de categorias como por ejemplo:
\begin{itemize}
	\item \textbf{Funcionales}: Cosas que no nos gustan de como funciona la maqueta, como podrian ser los semaforos, iluminacion, trazado de las vias,\dots Aqui entraria tambien descarrilamientos por poner mal las vias.
	\item \textbf{Esteticas}: Relacionado de como queda visualmente algo. Quizas poner casas de la montaña en zona llana caribeña no queda muy bien,\dots. Si vien la estetica es muy personal, pero si a las personas interesadas les molesta es un fallo\footnote{Si eres un visitante ocasional, podras comentarlo, pero que no coincida con tus gustos no siginifca que es un fallo}.
	\item \textbf{Realismo}: Esto es cuando la maqueta no representa algo que pueda suceder en la realidad. Podemos divir en muchas subcategorias, como de explotacion (mover los trenes como en la realidad), coherencia de epocas y lugares (mezclar materia español epoca II con aleman VI) o de coherencia escenica (mezclar edificios, no cooncordar trazado con elementos). Recordemos que los fallos solo son tales si las personas interesadas lo dicen. Especialemente en este caso, si la falta de realismo no molesta nunca puede ser un fallo. 
	
	\item \textbf{Otros}: Cada persona puede tener su propia clasificacion, y podriamos poner más, pero las quejas más oidas sobre maquetas suelen entrar en esas tres anteriores. Como por ejemplo podemos añadir aqui, el no cumplimento de objetivos planteados para la maqueta.
\end{itemize}

La causa del fallo, la consideramos practicamente irrelevante, en general va a ser por falta de habilidad en el momento de ejecucion (y se solucionara con practica o con cuidado) o por falta de informacion durante la ejecucion o que es lo mismo hemos evolucionado y tenemos nueva formacion y por ello detectamos un fallo. 

El efecto, lo podemos resumir por cuanto nos molesta dicho fallo.

Y el cuando segun se introduce el fallo, ya sea en diseño, construccion o explotacion. Dicha esta clasificación rapida, para el ejercicio mental nos vamos a quedar unicamente la idea de cuando se crea el fallo. Por que al final, los fallos van a existir independientemente de su razon y efecto. Añadiremos un cuarto momento quedando los siguientes tipos:
\begin{itemize}
	\item \textbf{Fallos de diseño}: Son aquellos que se crean cuando se esta pensando la maqueta y, para el ejercicio, se tenia informacion para evitarlo. Por ejemplo un trazado que nos aburre o una playa de vias que no permite manejar el material que queremos. 
	\item \textbf{Fallos de diseño por evolucion}: Categoria añadida para el ejercicio, Son los que se crean en el diseño, pero los vemos cuando hemos aprendido cosas nuevas o refinado nuestros gustos. Por ejemplo la playa de vias no nos sirve con el material que hayamos adquirido a posteriori. O por que hemos leido un libro que explique algo que no tuvimos en cuenta en su momento\footnote{Si ya teniamos el libro en el momento de diseñar es de la categoria anterior}.
	\item \textbf{Fallos de construccion}: Son los que se producen durante la creacion pero no es causado por el diseño. Lo más facil de entender es el ejemplo de poner mal las vias, pero tambien entraria tener que mover un desvio por que justo coincide con un nervio de la estructura y ya no nos convence su funcionomiento/posicion.
	\item \textbf{Fallos de explotacion}: Esta es la categoria par indicar que los fallos son daños que ha recibido la maqueta por la razon que sea (mudanza, mascotas, algo que se cae encima, polvo que no se puede quitar,\dots).
\end{itemize}

Es importante separar los fallos de diseño por evolucion de los que no, por que mentalmente racionalizamos los primeros con el conocimiento, \emph{tenemos este fallo, pero no teniamos forma de evitarlo por que no sabiamos...} en contra \emph{este fallo lo podriamos haber evitado por que lo sabiamos}


De estos cuatro tipos de fallos, para el ejercicio, los fallos de explotacion, son los más aletorios y dificiles de preever por que se presuponen accidentales al igual que de los fallos de diseño por evolución, no van a depender de como se haga la maqueta del resto lo vemos como en los apartado anteriores:


\begin{multicols}{3}
	\textbf{Montaje a lo loco}
	
	Se cometeran muchos errores de diseño, puesto que no hay diseño. Igualmente al ir haciendo y deshaciendo sera muy facil introducir fallos de construccion.
	
	\columnbreak
	
	\textbf{Montaje Pensado}
	
	En este caso, no habra muchos fallos de diseño, si lo hay. Al haber pensado las cosas tampoco habra muchos fallos de construccion.
	
	Pero cuando evolucionemos y veamos fallos, es probable que no podamos separa los fallos de diseño por evolucion de los de diseño real. Con la consiguiente carga mental aumentando el descontento con dicho fallo.
	
	\columnbreak
	
	\textbf{Montaje siguiendo un proceso}
	
	Seguir un proceso de ingenria no garantiza que no haya fallos de diseño, pero si que van a ser minimos, menos que el anterior. Y lo mismo para los de construccion, nos minimiza pero no elimina todos, sobre todo ya que este tambien depende de nuestra habilidad de construccion.
	
	Pero si, cuando veamos un fallo, podremos ver el diseño en la documentacion que hayamos hecho y la clasificacion, sera más facil ver si se deben a evolucion o no. Y en el caso de serlo no nos cargara tanto.
	
\end{multicols}

\section{Resultados o Datos de interés}(Opcional)
Si es un experimento incluir los datos o resultados obtenidos, sin valorar ni judgar. Es buen lugar para incluir otros detalles encontrados durante la escritura, búsqueda de información,....
\section{Discusión}
Este el punto para valorar los resultados y dar opiniones.
\section{Conclusiones}
Resumir y agrupar los resultados obtenidos
\section{Próximos pasos}
Escribir aquí un breve texto de lo que se hablara en otros capítulos (y que tenga referencia con este), o cosas que se dejan para realizar en un futuro fuera de este PDF.
\section{Bibliografía y Referencias}
\printbibliography[heading=subbibliography]