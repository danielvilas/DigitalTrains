% !TeX encoding = UTF-8
% !TeX spellcheck = es_ES
% !TeX root = ../../main.tex

\epigraph{El mundo entero se aparta cuando ve pasar a un hombre que sabe a dónde va}{Antoine de Saint-Exupéry }

\begin{abstract}
    En un proyecto de ingeniería el primer paso es la captura de requisitos.
    Seguramente un perfil mas comercial dirá que el primero es venderlo, pero para venderlo hay que dar un precio
    y para dar ese precio hay que estimar cuanto costara, para ello se necesita saber lo que se quiere,
    ergo los requisitos diría el perfil más técnico, luego seguramente acabarían comentando que si toma o captura
    de requisitos. Según como sea podrán estar así horas y horas para decir lo mismo con distintos términos.
\end{abstract}

\section{Introducción}

En la realidad hay dos momentos donde se recogen los requisitos, en una fase pre-venta se recogen a alto nivel
(por ejemplo que el puente soporte el peso de 10 coches) y una vez contratado el proyecto se realiza otra fase con
más detalle (10 coches coches se convierten en X toneladas suponiendo la media de los que pasan es Y más un margen
de Z\%,…).
Estos dos momentos se suelen llamar toma y captura, pero cada metodología y/o empresa define cual de esos términos
es cada fase, incluso cambiando el nombre.

Es importante realizar esta captura lo antes posible y con el mayor nivel de detalle posible.
Cambiar los requisitos (modificarlos, quitar o añadir alguno) significa revisar todas las decisiones y acciones
realizadas hasta el momento que se puedan ver afectadas por dicha modificación.
Por lo que si se debe realizar los cambios, costara más cuanto más nos acercaremos al final.

Por ejemplo, si inicialmente tenemos un espacio de 3x4 metros y luego cambiamos a 2x3 metros.
No es lo mismo que estemos pensando el diseño de la maqueta, a tener ya las vías pegadas y empezando a hacer
la decoración escénica. Ya que esta es un requisito que afecta a muchas acciones y decisiones.
Por otra parte, otros requisitos pueden cambiarse casi sin afectar.
Imaginémonos que en un principio queremos representar una escena invernal, pero antes de hacer el paisaje se
decide cambiar por una veraniega y el impacto será mínimo puesto que aun no se había empezados.
Ojo, que quizás alguna decisión pudiera ser sido diferente, como podría ser añadir una playa con un puerto
si se hubiera decidido antes.

\section{Estado del arte}
Centrándonos en que esto es para una maqueta y no un proyecto de ingeniería al uso,
no necesitamos toda una disertación de fases, costes, análisis, gestión del cambio, impacto, etc.
Nos tenemos que quedar con la idea de definirlos lo antes posible, puesto que cambiarlo más tarde costara.
Es decir necesitamos tener el listado de requisitos y pensados con cabeza.

Para poder hablar de los requisitos debemos primero, ver lo que son y sus características. Despues en la seccion
de discusion hablaremos en más detalle\footnote{NdA: Siguiendo una estrucutra más academica}.

\subsection{¿Qué son los requisitos?}

En resumen los requisitos son únicamente el listado de cosas que tiene que cumplir la maqueta, o los objetivos
de la misma. En un proyecto de ingenierías se suelen dividir en tipos, como funcionales (Lo que tiene que hacer)
y técnicos (restricciones físicas) para la proyectos informáticos.
Cada ingeniería tiene su propia categoría y aquí estamos en un hobby de maquetas.
Por tanto los organizaremos y clasificaremos por lo que más nos interese.
Si que podría ser interesante tenerlos organizados por intencionalidad (que se quiere lograr),
escénico (tener tal y tal cosa) y físicos (donde debe caber u otras limitaciones realacioandos con espacio).

Es muy importante tener en cuenta que los requisitos son lo que luego que dictaminarán si una maqueta es un
éxito y por tanto buena. Dicho de otra forma, será una buena maqueta si cumple los requisitos u objetivos
planteados. Así que dichos requisitos deben estar escritos en algún sitio.

Los requisitos siempre responden a preguntas del tipo ¿Qué quiero…? ¿Dónde quiero…?
O ¿Qué debe …? O ¿Dónde debe…?

Algunos requisitos estarán muy relacionados entre si, quizás siendo aclaraciones o detalles de uno.
Por lo que podremos considerarlos como Sub-requisitos.

\subsection{Objetivos y Prioridades}

Los objetivos son los requisitos que nos parecen más importantes y son lo que consideramos básicos que
la maqueta debe cumplir. Son particulares de cada maquetista y ninguno es trivial. Básicamente responden
¿Qué quiero conseguir con mi maqueta? Aunque al final serán los que se consideren más importantes.

Como se puede intuir, no todos los requisitos será igual de importantes, y los menos importantes podremos
no cumplirlos o modificarlos para que se ajusten a lo que tenemos creado. Pero los más importantes deberemos
mantenerlos lo más fijos posibles.

\subsection{Detalle de los requisitos}

Los requisitos deberían ser breves, concisos, concretos y claros. Pero a su vez deben tener el mayor detalle
posible para que sea lo más fácil posible tomar las decisiones futuras. Por ejemplo veamos posibles requisitos,
que son el mismo, pero con diferente nivel de detalle.

\begin{enumerate}
    \item Quiero un parque de atracciones.
    \item Quiero un parque de atracciones inspirado en el de mi ciudad
    \item Quiero un parque de atracciones con una noria y un lago.
    \item Quiero un parque de atracciones con una noria modelo tal y un lago que tenga 3 barcas modelo tal.
    \item Quiero un parque de atracciones inspirado en el de mi cuidad, con al menos las atracciones … y siguiendo el mapa …
\end{enumerate}

Como podemos ver el número 1 es el más genérico y abstracto, pero también nos permite más flexibilidad en
decisiones futuras. Por otra parte el 4 y el 5 son los más concretos y con más detalle, son más rígidos
pero por otra parte nos fija cosas que luego nos evitamos pensar.

En el caso del 5, incluso convendría, partir el listado de atracciones en Sub-requisitos para que no quede muy
largo el mismo, pero considerarlos todos como uno, si va a ser un objetivo de la maqueta.

\subsection{Cuando tener los requisitos}

Los requisitos deben estar lo antes posible y siempre antes de que comiencen a necesitarse.
Los requisitos serán la plantilla para la toma de decisiones como la elección entre alternativas.
Por lo tanto no es necesario tener todos al principio, pero si los objetivos.

Volviendo al ejemplo anterior, suponiendo que hacemos una maqueta por módulos y vamos a tratar el modulo
donde pondremos el parque de atracciones. Realmente hasta este momento no hemos necesitado el requisito hasta
este momento, por lo tanto podremos no tenerlo o modificarlo ”sin coste” hasta ahora.
Pero si lo modificamos después de hacer el modulo, puede que tengamos que rehacerlo.

Es decir necesitaremos tener el requisito lo más detallado junto antes de usarlo, pero en el listado de
requisitos debería estar, aunque sea con baja definición, desde el momento se nos pase por la cabeza.

En el ejemplo del parque de atracciones, tendríamos el 1 al principio, más adelante cuando se planifiquen
los módulos, algo un poco más detallado como el 2 o 3, para dar una idea más clara.
Pero el momento de diseñar el modulo concreto el 4 o 5.
\subsection{Requisitos de espacio}

\section{Ejemplo Practico}
\section{Texto principal}

\section{Resultados}
%\subsection{Los requisitos de una maqueta}

La maqueta que queremos desarrollar será sobre una empresa ficticia DanielBahn y el resultado final será
el resultado de varios años de trabajo, en este momento y estos artículos se describe el proceso para un
maqueta más pequeña DanielTeppichBahn\footnote{TeppichBahn es como llaman en Alemania a las maquetas de
    tren que se montan sobre el suelo para jugar y se desmontan cuando ya se ha acabado el juego.}.
Aunque, teniendo que esta maqueta será para probar técnicas/tecnologías para la versión final.
Se tendrán en cuenta en los requisitos.

El proceso para tener los requisitos es iterativo, es decir se van escribiendo en iteraciones,
intentando ir ampliando poco a poco para cada maqueta.

Para DanielBahn tenemos los objetivos  siguientes:
\begin{itemize}
    \item Representar tres escenas inspiradas en sitios reales, por importancia sentimental
          \begin{itemize}
              \item La estación del puerto de La Coruña hasta la playa de lazareto
              \item La estación del Santo Sepulcro de Zaragoza
              \item Una estación de montaña como la de Canfranc
          \end{itemize}
    \item Desarrollar módulos electrónicos en LCC para desvíos, y paneles de control.
    \item DCC para las vias y para los mandos, cualquiera con soporte para JMRI
    \item Tener un una vía continua para realizar fotos, sin preocupaciones de tener que evitar choques contra fin de vía
    \item Tener una zona de puzzle de clasificación
    \item Ser más o menos realista en trazado y operación.
\end{itemize}


Como se puede ver son lo suficientemente concretos para poder ir diseñando y pensando alternativas,
pero tan genéricos que dan lugar multitud de opciones. Así mismo es tan a largo plazo, que es más una lista de
deseos que un listado de requisitos al uso. Tampoco hay restricciones de espacio, a la espera de realizar diseños
y buscar alternativas.

DanielTeppichBahn, por su parte será una maqueta para poner en el suelo, y de aprendizaje por los que sus
objetivos son:
\begin{itemize}
    \item Tener una maqueta pequeña para:

          \begin{itemize}
              \item Rodar los trenes
              \item Aprender técnicas
              \item Probar nuevas ideas
          \end{itemize}
    \item Debe caber “escondida” detrás de un mueble:
    \item Debe ser fácil de montar
    \item Con Electrónica para controlar desvíos, y luces hecha por mi.
    \item Debe tener los problemas típicos(Si es posible):

          \begin{itemize}
              \item Playa de vías
              \item Vías de escape
              \item Cruces
              \item Bucles
          \end{itemize}
    \item DCC para las vías y los desvíos
    \item Loconet para los mandos, módulos propios para JMRI
    \item Paneles y módulos propios en Loconet (para ahorrar cables)
    \item Basada en montaje de caja inicial
\end{itemize}

Como se puede apreciar, en esta maqueta ya aparece un requisito donde debe estar guardada, porque es el compromiso
que hemos podido llegar, para tener una maqueta. Y son más concretos, con lo que nos permite ir desarrollando
esta maqueta.

Seguramente, iremos desarrollando maquetas según el tiempo y el espacio disponible vayan variando.
\section{Discusión}
\subsection{Requisitos de espacio}

Las limitaciones de espacio, y por ende sus requisitos, se suelen tener en cuenta antes si quiera de empezar
a pensar en la maqueta. Esto es un reflejo de las situación de cuando podemos hacer la maqueta.
Normalmente cuando tenemos ya la vida resuelta y el espacio ocupado ya en la casa por estanterías, muebles
y demás.

Esto nos obliga a hacer compromisos, con otras personas, el espacio y lo que queríamos hacer, por lo que
muchas veces acabamos con una maqueta que no nos satisface del todo. Desde este apartado, abogamos por
retrasar estos requisitos lo máximo posible hasta una fase de análisis y diseño con los requisitos que se
realmente se quiere.

Para poder retrasar estos requisitos necesitamos una fase de análisis y diseño, donde veremos el espacio
que requiriéramos para el resto de requisitos y nos obligará a pensar alternativas de espacio
(alquilar un local, usar la casa del pueblo, mudarse, …), pensar en diseños para varias habitaciones, o
un cambio de diseño (pasar a escenas en varios niveles, segmentada, de maleta,…).

También nos permite probar varios diseños, sobre el papel o sobre pruebas, donde llegaremos a compromisos,
con la seguridad que hemos intentando todo lo posible por tener todo aquello que queríamos necesariamente.
Y que ademas ha sido imposible tener más.

Esto no quiere decir que tener requisitos de espacio al principio sea malo. Los tendremos si es algo ”fijo”
puesto que se ha acordado previamente con otras personas del hogar, o porque es la única posibilidad real.
Pero recordemos que luego un cambio de esto, tiene grandes implicaciones, por lo que cuanto más tarde 
lo tengamos mejor. Aunque implicara pensar más alternativas y por lo tanto más trabajo intelectual.

\section{Conclusiones}
Como hemos comentado, lo importante de los requisitos es tener un documento o listado de lo que se quiere
tener. Este listado debe ser lo mas detallado posible, pero cuando se necesite. Al principio lo tendremos 
con poco detalle y lo iremos detallando a necesidad. Nos preocuparemos más de lo que queremos y las
limitaciones intentaremos retrasarlas para buscar alternativas.

También hemos presentado los objetivos de la maqueta final, a largo plazo (DanielBahn) y otra mas cercana
en el tiempo para ir aprendiendo y tener un sitio donde rodar (DanielTeppichBahn)

Como conclusión, tengamos sentido común, y hagamos una lista de lo que realmente queramos y luego
estudiemos opciones para ver como podemos llegar a ellas. No tengamos miedo a pensar a largo plazo y
hacer maquetas intermedias para aprender, o probar cosas. 
\section{Próximos pasos}

\section{Bibliografía}
\printbibliography[heading=subbibliography]
