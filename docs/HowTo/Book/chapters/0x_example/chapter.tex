% !TeX encoding = UTF-8
% !TeX spellcheck = es_ES
% !TeX root = ../../main.tex


\epigraph{Los hombres no crecen, solo cambian el precio y tamaño de sus juguetes}{Cita anonima en Internet}

\begin{abstract}
Hay varias formas de jugar con una maqueta de tren, en este capitulo revisaremos algunas de las más comunes
\end{abstract}

\section{Introducción}
Incluir aquí una introducción al capitulo, estableciendo un contexto para centrar al lector
\cite{ackerberg2006} Es una prueba solo para jugar

\section{Estado del arte}
Explicar como esta actualmente el hobby o las diferentes publicaciones respecto al tema
\section{Experimento o Texto principal}
Describir de la manera más aseptica posible lo que se quiere avanzar
\section{Resultados o Datos de interés}(Opcional) 
Si es un experimento incluir los datos o resultados obtenidos, sin valorar ni judgar. Es buen lugar para incluir otros detalles encontrados durante la escritura, búsqueda de información,....
\section{Discusión}
Este el punto para valorar los resultados y dar opiniones.
\section{Conclusiones}
Resumir y agrupar los resultados obtenidos
\section{Próximos pasos}
Escribir aquí un breve texto de lo que se hablara en otros capítulos (y que tenga referencia con este), o cosas que se dejan para realizar en un futuro fuera de este PDF.
\section{Bibliografía y Referencias}
\printbibliography[heading=subbibliography]